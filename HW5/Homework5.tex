% --------------------------------------------------------------
% This is all preamble stuff that you don't have to worry about.
% Head down to where it says "Start here"
% --------------------------------------------------------------
 
\documentclass[12pt]{article}
 
\usepackage[margin=1in]{geometry} 
\usepackage{amsmath,amsthm,amssymb,scrextend}
\usepackage{fancyhdr}
\usepackage{enumitem}
\usepackage{amsmath}
\usepackage{amssymb}
\usepackage{textcomp}
\usepackage{fancybox}
\usepackage{tikz}
\usepackage{tasks}
\pagestyle{fancy}
\usepackage[makeroom]{cancel}
\usepackage{graphicx}
\usepackage{caption}
\usepackage{mwe}
\usepackage{tikz}
\usetikzlibrary{positioning}

\newcommand{\N}{\mathbb{N}}
\newcommand{\Z}{\mathbb{Z}}
\newcommand{\I}{\mathbb{I}}
\newcommand{\R}{\mathbb{R}}
\newcommand{\Q}{\mathbb{Q}}
\renewcommand{\qed}{\hfill$\blacksquare$}
\let\newproof\proof
\renewenvironment{proof}{\begin{addmargin}[1em]{0em}\begin{newproof}}{\end{newproof}\end{addmargin}\qed}
% \newcommand{\expl}[1]{\text{\hfill[#1]}$}
 
\newenvironment{theorem}[2][Theorem]{\begin{trivlist}
\item[\hskip \labelsep {\bfseries #1}\hskip \labelsep {\bfseries #2.}]}{\end{trivlist}}
\newenvironment{lemma}[2][Lemma]{\begin{trivlist}
\item[\hskip \labelsep {\bfseries #1}\hskip \labelsep {\bfseries #2.}]}{\end{trivlist}}
\newenvironment{problem}[2][Problem]{\begin{trivlist}
\item[\hskip \labelsep {\bfseries #1}\hskip \labelsep {\bfseries #2.}]}{\end{trivlist}}
\newenvironment{exercise}[2][Exercise]{\begin{trivlist}
\item[\hskip \labelsep {\bfseries #1}\hskip \labelsep {\bfseries #2.}]}{\end{trivlist}}
\newenvironment{reflection}[2][Reflection]{\begin{trivlist}
\item[\hskip \labelsep {\bfseries #1}\hskip \labelsep {\bfseries #2.}]}{\end{trivlist}}
\newenvironment{proposition}[2][Proposition]{\begin{trivlist}
\item[\hskip \labelsep {\bfseries #1}\hskip \labelsep {\bfseries #2.}]}{\end{trivlist}}
\newenvironment{corollary}[2][Corollary]{\begin{trivlist}
\item[\hskip \labelsep {\bfseries #1}\hskip \labelsep {\bfseries #2.}]}{\end{trivlist}}
 
\setlength{\parindent}{0pt}
\begin{document}
 \settasks{
	counter-format=(tsk[r]),
	label-width=4ex
}
% --------------------------------------------------------------
%                         Start here
% --------------------------------------------------------------

\lhead{Math 632}
\chead{Homework 5}
\rhead{Meenmo Kang}

\noindent
\textbf{Question 1}\\
Copy machine 1 is in use now. Machine 2 will be turned on at time t.
Suppose that the machines fail at rate $\lambda_i$. What is the probability that machine 2 is the first to fail?

\vspace{1.5\baselineskip}

\textbf{Question 2}\\
Let $S$ and $T$ be exponentially distributed with rates $\lambda$ and $\mu$. Let $U= min\{S,T\}$ and $V=max\{S,T\}$. Find
\begin{enumerate}[label=(\alph*)]
    \item $EU$
    \item $E(V-U)$. Compute first $P(V-U > s)$ for $s>0$ either by integrating densities of $S$ and $T$ or by conditioning on the events $S<T$ and $T<S$. From $P(V-U>s)$ deduce the density function $f(v-u)$ of $V-U$, and then the mean $E(V-U)$ by integrating the density.
    \item $EV$
\end{enumerate}
Finally, check that your answers to (a),(b),(c) satisfy $E(V-U) = E(V)-E(U)$.


\vspace{1.5\baselineskip}
\textbf{Question 3}\\
In a hardware store you must first go to server 1 to get your goods and then go to a server 2 to pay for them. Suppose that the times for the two activities are exponentially distributed with means 6 minutes and 3 minutes. Find the answer when times for the two activities are exponentially distributed with rates $\lambda$ and $\mu$. For symmetric notation, let $A_i$ denote the amount of time Al spends at server $i$, and $B_i$ the amount of time Bob spends at server $i$, for $i=1,2$.\\
$Hint$. Draw a picture of the time line to understand how Al and Bob move through the servers. Use your answer from 2.7. This problem requires no integration.

\vspace{1.5\baselineskip}
\textbf{Question 4}\\
A machine has two critically important parts and is subject to three different types of shocks. Shocks of type i occur at times of a Poisson process with rate $\lambda_i$. Shocks of types 1 break part 1, those of type 2 break part 2, while those of type 3 break both parts. Let $U$ and $V$ be the failure times of the two parts.
\begin{enumerate}[label=(\alph*)]
    \item Find $P(U>s,V>t)$
    \item Find the distribution of $U$ and the distribution of $V$.
    \item Are $U$ and $V$ independent?
\end{enumerate}
$Hint.$ Let $N_i$ denote a Poisson process of rate $\lambda_i$. For part (a), express the event in terms of $N_1, N-2$ and $N_3$. Part (b) follows quickly from (a). For part (c), check whether $P(U>s,V>t)$ equals $P(U>s)\cdot P(V>t)$.

\vspace{1.5\baselineskip}
\textbf{Question 5}\\
Consider a Poisson process of rate $\lambda$ and let $s$ be a fixed positive number. Let $\sigma$ be the
random amount of time from s till the next arrival. In symbols,
$$\sigma = \left(\min\limits_{k:T_k>s} T_k\right) - s.$$
{\sl Calculate rigorously the probability} the probability $P(\sigma>a)$ for real $a>0$ by using the density functions of the arrival times $T_k$ and the interarrival times $\tau_k$.\\
$Hint.$ Draw a picture of the time line. Since $s+\sigma$ is one of the arrival times, you can decompose the probability $P(\sigma>a)$ into different cases according to which $T_k$ is equal to $s+\sigma$.


\end{document}