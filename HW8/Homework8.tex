% --------------------------------------------------------------
% This is all preamble stuff that you don't have to worry about.
% Head down to where it says "Start here"
% --------------------------------------------------------------
 
\documentclass[12pt]{article}
 
\usepackage[margin=1in]{geometry} 
\usepackage{amsmath,amsthm,amssymb,scrextend}
\usepackage{fancyhdr}
\usepackage{enumitem}
\usepackage{amsmath}
\usepackage{amssymb}
\usepackage{textcomp}
\usepackage{fancybox}
\usepackage{tikz}
\usepackage{tasks}
\pagestyle{fancy}
\usepackage[makeroom]{cancel}
\usepackage{graphicx}
\usepackage{caption}
\usepackage{mwe}
\usepackage{tikz}
\usetikzlibrary{positioning}

\newcommand{\N}{\mathbb{N}}
\newcommand{\Z}{\mathbb{Z}}
\newcommand{\I}{\mathbb{I}}
\newcommand{\R}{\mathbb{R}}
\newcommand{\Q}{\mathbb{Q}}
\renewcommand{\qed}{\hfill$\blacksquare$}
\let\newproof\proof
\renewenvironment{proof}{\begin{addmargin}[1em]{0em}\begin{newproof}}{\end{newproof}\end{addmargin}\qed}
% \newcommand{\expl}[1]{\text{\hfill[#1]}$}
 
\newenvironment{theorem}[2][Theorem]{\begin{trivlist}
\item[\hskip \labelsep {\bfseries #1}\hskip \labelsep {\bfseries #2.}]}{\end{trivlist}}
\newenvironment{lemma}[2][Lemma]{\begin{trivlist}
\item[\hskip \labelsep {\bfseries #1}\hskip \labelsep {\bfseries #2.}]}{\end{trivlist}}
\newenvironment{problem}[2][Problem]{\begin{trivlist}
\item[\hskip \labelsep {\bfseries #1}\hskip \labelsep {\bfseries #2.}]}{\end{trivlist}}
\newenvironment{exercise}[2][Exercise]{\begin{trivlist}
\item[\hskip \labelsep {\bfseries #1}\hskip \labelsep {\bfseries #2.}]}{\end{trivlist}}
\newenvironment{reflection}[2][Reflection]{\begin{trivlist}
\item[\hskip \labelsep {\bfseries #1}\hskip \labelsep {\bfseries #2.}]}{\end{trivlist}}
\newenvironment{proposition}[2][Proposition]{\begin{trivlist}
\item[\hskip \labelsep {\bfseries #1}\hskip \labelsep {\bfseries #2.}]}{\end{trivlist}}
\newenvironment{corollary}[2][Corollary]{\begin{trivlist}
\item[\hskip \labelsep {\bfseries #1}\hskip \labelsep {\bfseries #2.}]}{\end{trivlist}}
 
\setlength{\parindent}{0pt}
\begin{document}
 \settasks{
	counter-format=(tsk[r]),
	label-width=4ex
}
% --------------------------------------------------------------
%                         Start here
% --------------------------------------------------------------

\lhead{Math 632}
\chead{Homework 8}
\rhead{Meenmo Kang}
\begin{enumerate}
\item Consider an M/M/1 queue (Durrett Example 4.16) where customers arrive at rate $\lambda$, and the service time for each server is a rate $\mu$ exponential random variable. Let X(t) denote the number of customers in the system at time $t$. Assume that $X(0) = 0$.
\begin{enumerate}[label=(\alph*)]
    \item State the Kolmogorov forward equations for this process.
    \item Set $M(t) = E[X(t)]$. Prove that
    $$\frac{dM}{dt} = \lambda - \mu M(t)$$
    {\sl Hint.} Use the equations from (a). Do not hesitate to dierentiate series term by term.
    
    \item Solve the differential equation for M(t). (A reminder about linear ODEs is appended to this HW sheet.)
    
    \item Evaluate $\lim\limits_{t\to\infty} M(t)$.The stationary distribution for $X(t)$ is given in Example 4.16 of Durrett's book. Compare the limit you found to the expected value of the stationary distribution.
\end{enumerate}
    
\item Consider an M/M/2 queue where customers arrive at rate $\lambda$ and the rate for each server
is $\mu$. However, arriving customers who see $N$ customers already in the system leave and
never return. Assume $N > 2$. Let $X(t)$ denote the number of customers in the system at
time $t$. Find the stationary distribution for $X(t)$. (The M/M/s queue appears in Durrett's
examples 4.3 and 4.17.)

\item A hemoglobin molecule can carry one oxygen or one carbon monoxide molecule. Suppose that the two types of gases arrive at rates 1 and 2 and attach for an exponential amount of time with rates 3 and 4, respectively. Formulate a Markov chain model with state space \{+, 0, \---\} where + denotes an attached oxygen molecule, \---− an attached carbon monoxide molecule, and 0 a free hemoglobin molecule and find the long-run fraction of time the hemoglobin molecule is in each of its three states.


\end{enumerate}



\end{document}